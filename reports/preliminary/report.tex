\documentclass{article}

\usepackage[utf8]{inputenc}
\usepackage[french]{babel}

\usepackage{hyperref}

\author{
    Valentin Jonquière,
    Mathilde Chollon,
    Denis Demirci,
    Iwen Jomaa,
    Jonathan Landry
}

\title{Rapport Préliminaire projet de programmation, Échecs en Java}

\begin{document}

\maketitle

\pagebreak

\tableofcontents

\pagebreak

\section{Contexte du sujet}

\section{Explication du sujet}

\section{Besoins visés}
\subsection{Quels besoins?}
Dans le sujet donné, il y avait une liste de 60 besoins.
Nous avons pris la décision ambitieuse de répondre à tous ces besoins. C'est un travail conséquent,
mais avec une équipe de 5 personnes, codant pendant huit semaines,
nous pensons que cet objectif est atteignable.

Nous avons tout de même classé ces besoins par ordre de priorité (voir \nameref{agenda}), afin de garantir
la jouabilité projet même si nous n'arrivons pas à tout implémenter à temps.
Nous voulons rendre un jeu cohérent et fonctionnel avec des modules entièrement
terminés (même s'il en manque) plutôt qu'un projet avec des règles manquantes,
une intelligence artificielle bâclée et une interface graphique peu fonctionnelle.

\subsection{Dépendances entre les besoins}
\subsubsection{Gestion des options}
La première dépendance que nous avons concerne la gestion des options
\subsubsection{Sauvegarde de l'historique}
La gestion de l'historique est un besoin indispensable pour notre jeu.
Beaucoup de besoins dépendent de cette fonctionnalité.
\subsubsection{Intelligence artificielle}
\subsubsection{Interface graphique}
\section{Agenda prévisionnel}
\label{agenda}

\section{Architecture visée pour le projet}

\section{Spécifications étendues}

\end{document}