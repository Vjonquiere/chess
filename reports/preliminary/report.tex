\documentclass{article}

\usepackage[utf8]{inputenc}
\usepackage[french]{babel}

\author{
    Valentin Jonquière,
    Mathilde Chollon,
    Denis Demirci,
    Iwen Jomaa,
    Jonathan Landry
}

\title{Rapport Préliminaire projet de programmation, Échecs en Java}

\begin{document}

\maketitle

\pagebreak

\tableofcontents

\pagebreak

\section{Contexte du sujet}
\subsection{Sujet choisi}
\subsection{Langage de programmation choisi}
Avec ce sujet, nous avions 3 choix de langage possibles :
\begin{itemize}
    \item Le langage C
    \item Le Java
    \item Le Python
\end{itemize}
Même si nous avons dû faire un choix rapidement, nous avons d'abord développé les
avantages et inconvenants de chacun de ces langages. De plus, nous nous étions mis
d'accord sur quelques besoins essentiels auquel certains des langages ci-dessus 
n'aurait pas (ou difficilement) pu répondre.
Tout d'abord nous souhaitions pouvoir développer notre logiciel sous forme de modules
complémentaires pour plusieurs raisons.
\begin{itemize}
    \item Nous avons convenu de baser les 8 semaines de code en se concentrant sur 
    des blocs de 2 semaines avec, pour chaque bloc, une liste de 2 ou 3 modules à développer.
    \item Isoler les bugs dans chaque module afin d'avoir moins de problèmes lors de la fusion
    des modules.
    \item Facilité de maintien. Avec 8 semaines de code, il est essentiel que nous possédions
    un logiciel organisé afin de ne pas avoir à `fouiller' dans le code écrit la première semaine
    lors de la finalisation du projet.
    \item Utilisation de la programmation objet. Même s'il n'y a pas d'extension de prévue,
    ce paradigme est le plus simple et adopté lors du développement d'un jeu.
\end{itemize}
\subsection{Langage C}
La première option qui nous était proposée été le langage C. Le projet contenant des besoins posant
des contraintes de performances (cf. besoin), ce langage a donc semblé être une bonne idée, car il est
de loin celui permettant d'obtenir un logiciel performant. Cependant, celui-ci ne proposant pas de 
possibilité de développement objet, nous aurions dû faire une grosse concession dès le choix du langage.
De plus, la gestion de la mémoire étant à la charge du développeur, nous aurions pu rester bloqué un temps
précieux sur un pointeur mal alloué. Cette gestion de la mémoire apporte également le problème des fuites
mémoire qui demandent, elle aussi, parfois beaucoup de travail pour être résolues.

\section{Explication du sujet}

\section{Besoins visés}

\section{Agenda prévisionnel}

\section{Architecture visée pour le projet}

\section{Spécifications étendues}

\end{document}